\documentclass{beamer}
\usetheme{PSU}
%%%%%%%%%%%%%%%%%%%%%%%%%%%%%
%Author: Mahdi Belbasi
%Email: Belbasi@psu.edu
%Homepage: https://sites.google.com/view/belbasi
%%%%%%%%%%%%%%%%%%%%%%%%%%%%%


%\setbeameroption{show notes}
%\setbeameroption{show only notes}
\setbeamerfont{note page}{size=\fontsize{0.2cm}{0.1cm}}

\usepackage{media9,tagging}
\usepackage{pgfplots}
\usepackage{amsmath}
\pgfplotsset{compat=1.10}
\usetikzlibrary{calc}
\usepackage[utf8]{inputenc}
\usepackage[T1]{fontenc}

%% Use any fonts you like.
\usepackage{helvet,multicol}

%Idioma español
\usepackage[utf8]{inputenc}
\usepackage[spanish,es-tabla]{babel}
\newcommand\Newline{\newline\newline}

\title{\textcolor{white}{Tema 40}}
\subtitle{\textcolor{white}{Técnicas de mecanizado, conformación y unión de piezas metálicas.}}



\begin{document}

\begin{frame}[plain,t]
\titlepage
\end{frame}



\section{Técnicas de mecanizado}
\begin{frame}{Técnicas de mecanizado de piezas metálicas. Herramientas y útiles característicos}
    El mecanizado de materiales se hace con ayuda de uns máquinas llamadas máquinas-herramienta.
    A la operación de rebajado de material de una pieza con una herramienta cortante se le llama mecanizado. 
    

\end{frame}


\begin{frame}{Técnicas de conformación de piezas metálicas}
  \begin{itemize}
      \item Aserrado
      \item Corte oxiacetilénico
      \item Corte con láser
      \item Corte con plasma
      \item Torneado
      \item Taladrado
      \item Limado
      \item Cepillado
      \item Fresado
      \item Rectificado
      \item Brochado
      \item Mandrinado
  \end{itemize}
    
    \end{frame}

    \section{Técnicas de conformación}

\subsection{Conformación por moldeo}
\begin{frame}{Conformación por moldeo}
    \begin{itemize}
        \item Moldeo en arena
        \item Moldeo en coquilla
    \end{itemize}
\end{frame}

\subsection{Conformación por deformación}
\begin{frame}{Conformación por deformación}
    \begin{itemize}
        \item Forja
        \item Estampación
        \item Laminación
        \item Extrusión
        \item Estirado
        \item Trefilado
    \end{itemize}
\end{frame}

\section{Técnicas de unión}
\subsection{Uniones fijas}
\begin{frame}{Técnicas de unión: uniones fijas}
    \begin{itemize}
        \item Soldadura
       \begin{itemize}
        \item Soldadura blanda
        \item Soldadura fuerte
        \item Soldadurapor arco eléctrico
        \item Soldadura por resistencia
        \item Soldadura oxiacetilénica
        \item Soldaduras en frío
        \end{itemize}
        \item Remachado
        \item Unión por adhesivos
        \item Unión a presión 

    \end{itemize}
\end{frame}


\subsection{Uniones desmontables}
\begin{frame}{Técnicas de unión: Uniones desmontables}
\begin{itemize}
    \item Elementos roscados
    \begin{itemize}
        \item Tornillos y tuercas
        \item Pernos
        \item Trasfondos
        \item Prisioneros
        \item Espárragos
    \end{itemize}
    \item Elementos no roscados
    \begin{itemize}
        \item Pasadores
        \item Chavetas
        \item Guías
    \end{itemize}
\end{itemize}

\end{frame}
\ThankYouFrame

\end{document}