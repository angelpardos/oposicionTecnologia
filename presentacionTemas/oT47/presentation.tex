\documentclass{beamer}
\usetheme{PSU}
%%%%%%%%%%%%%%%%%%%%%%%%%%%%%
%Author: Mahdi Belbasi
%Email: Belbasi@psu.edu
%Homepage: https://sites.google.com/view/belbasi
%%%%%%%%%%%%%%%%%%%%%%%%%%%%%


%\setbeameroption{show notes}
%\setbeameroption{show only notes}
\setbeamerfont{note page}{size=\fontsize{0.2cm}{0.1cm}}

\usepackage{media9,tagging}
\usepackage{pgfplots}
\usepackage{amsmath}
\pgfplotsset{compat=1.10}
\usetikzlibrary{calc}
\usepackage[utf8]{inputenc}
\usepackage[T1]{fontenc}

%% Use any fonts you like.
\usepackage{helvet,multicol}

%Idioma español
\usepackage[utf8]{inputenc}
\usepackage[spanish,es-tabla]{babel}
\newcommand\Newline{\newline\newline}

\title{\textcolor{white}{Tema 47}}
\subtitle{\textcolor{white}{Máquinas térmicas: funcionamiento, clasificación y aplicaciones.}}



\begin{document}

\begin{frame}[plain,t]
\titlepage
\end{frame}



\section{Conceptos físicos}
\begin{frame}{Conceptos físicos}
    \begin{itemize}
        \item Calor y trabajo
        \item Potencia
        \item Energía
        \item Temperatura
        \item Calor específico
    \end{itemize}
  
\end{frame}

\section{Transformaciones termodinámicas}
\begin{frame}{Transformaciones termodinámicas}
    \begin{itemize}
        \item Primer principio de la termodinámicasTrabajo en una transformación cíclica
        \item Transformaciones de un sistema gaseoso
        \item Transformación isobárica
        \item Transformación isócora
        \item Transformación isoterma
        \item Transformación politrópica de índice n
        \item Segundo principio de la termodinámica
        \begin{itemize}
            \item Diagrama esquemático de una máquina térmicasRendimiento de una máquina térmicas
        \end{itemize}
        \item Ciclo de carnot
    \end{itemize}
\end{frame}


\section{Máquinas térmicas}
\begin{frame}{Máquinas térmicas}
    \begin{figure}[h]
        \includegraphics[width=8cm, height=8cm]{Selección_011.png}
        \end{figure}
\end{frame}


\begin{frame}{Máquinas térmicas: motores}
    \begin{itemize}
        \item Motor alternativo de combustión interna
        \item  \begin{itemize}
            \item Tipos de ciclos de funcionamiento
            \item Tipos de ciclos termodinámicos
        \end{itemize}
        \item Motor rotativo de combustión interna
        \item \begin{itemize}
            \item Motor alternativo de combustión externa
            \item Motor rotativo de combustión externa
        \end{itemize}
    \end{itemize}
  
\end{frame}



\section{Máquinas térmicas: generadores}
\begin{frame}{Máquinas térmicas: generadores}
    \begin{itemize}
        \item Máquina frigorífica ideal
        \item Máquina frigorífica real
        \item Bomba de calor
    \end{itemize}
  
\end{frame}



\ThankYouFrame

\end{document}