\documentclass{beamer}
\usetheme{PSU}
%%%%%%%%%%%%%%%%%%%%%%%%%%%%%
%Author: Mahdi Belbasi
%Email: Belbasi@psu.edu
%Homepage: https://sites.google.com/view/belbasi
%%%%%%%%%%%%%%%%%%%%%%%%%%%%%


%\setbeameroption{show notes}
%\setbeameroption{show only notes}
\setbeamerfont{note page}{size=\fontsize{0.2cm}{0.1cm}}

\usepackage{media9,tagging}
\usepackage{pgfplots}
\usepackage{amsmath}
\pgfplotsset{compat=1.10}
\usetikzlibrary{calc}
\usepackage[utf8]{inputenc}
\usepackage[T1]{fontenc}

%% Use any fonts you like.
\usepackage{helvet,multicol}

%Idioma español
\usepackage[utf8]{inputenc}
\usepackage[spanish,es-tabla]{babel}
\newcommand\Newline{\newline\newline}

\title{\textcolor{white}{Tema 47}}
\subtitle{\textcolor{white}{Máquinas térmicas: funcionamiento, clasificación y aplicaciones.}}



\begin{document}

\begin{frame}[plain,t]
\titlepage
\end{frame}



\section{Estructura de un automatismo}
\begin{frame}{Estructura de una automatismo}

\end{frame}


\section{El contactor}
\begin{frame}{El contactor}
    El contactor es un aparato mecánico de conexión accionado por un electroimán que funciona en todo o nada. Cuando la bobina del electroimán recibe corriente el contactor, mediante sus contactos, cierra el circuito entre la red de alimentación y el receptor. La parte móvil del electroimán que acción los contactos móviles de los polos y de los contactos auxiliares se desplaza de tres formas posibles: por rotación, 
    por translación y por movimiento combinado de los dos.
\end{frame}

\begin{frame}{El contactor}
    \begin{itemize}
        \item Constitución\begin{itemize}
            \item El electroimán
            \item Los polos
            \item Los contactos auxiliares
        \end{itemize}
        \item Comportamiento del cirucito magnético\begin{itemize}
            \item Relación entre ersfuerzo de atraccióny corriente de mando
            \item Arco eléctrico
            \item Parámetros para la elección de un contactor
        \end{itemize}
    \end{itemize}

\end{frame}


\section{Relés de uso industrial}
\begin{frame}{Relés de uso industrial}
\begin{itemize}
    \item Constitución
\end{itemize}
\end{frame}


\section{Protección de equipos eléctricos}
\begin{frame}{Protección de equipos eléctrico. El electroimán}
    \begin{itemize}
        \item Relés de protección
        \begin{itemize}
            \item Magnetotérmicos
            \item Relés térmicos
            \item Interruptor diferencial
            \item Sondas de termistenciaSeccionadores
            \item Interruptores de seguridad
        \end{itemize}
        \item Sondas de termistencia
        \item Fusibles
        \item Seccionadores
        \item Interruptores de seguridad
        \item Relés de medida
    \end{itemize}

\end{frame}


\section{Elementos auxiliares de mando}
\begin{frame}{Elementos auxiliares de mando}
    \begin{itemize}
        \item Auxiliares de mando manual
        \item Auxiliares de mando automático
        \begin{itemize}
            \item Contactos de mando mecánico.
            \item Detectores estáticos de proximidad
            \item Detectores fotoeléctricos
            \item Control de nivel.Interruptores de flotador
            \item Control de presión. Presostatos y vacuostatos.
            \item Conmutadores cíclicos
        \end{itemize}
    \end{itemize}


\end{frame}


\section{Representación simbólica de elementos}
\begin{frame}{Representación simbólica de elementos}
    \begin{itemize}
        \item Contactos
        \item Órganos de mando y medidaIdentificación de elementos
    \end{itemize}

\end{frame}

\section{Función memoria}
\begin{frame}{Función memoria}

\end{frame}

\section{Circuitos característicos}
\begin{frame}{Circuitos característicos}

\end{frame}




\ThankYouFrame

\end{document}